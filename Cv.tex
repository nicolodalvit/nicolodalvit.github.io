% LaTeX file for resume 
% This file uses the resume document class (res.cls)

\documentclass{res} 
\usepackage{ragged2e}
\usepackage{hyperref}
\usepackage{lipsum}

%\usepackage{helvetica} % uses helvetica postscript font (download helvetica.sty)
%\usepackage{newcent}   % uses new century schoolbook postscript font 
\setlength{\textheight}{9.5in} % increase text height to fit on 1-page 
\newcommand{\forceindent}{\leavevmode{\parindent=1em\indent}}

\begin{document} 

\fontsize{10}{12}\selectfont


\name{\centerline{Nicol\`{o} Dalvit}\\
\centerline{\href{mailto:nicolo.dalvit@sciencespo.fr}{nicolo.dalvit@sciencespo.fr}}\\
\centerline{\href{https://nicolodalvit.github.io/}{https://nicolodalvit.github.io/}}
\\
\noindent\makebox{\rule{\textwidth}{0.4pt}}
\\[12pt]}     % the \\[12pt] adds a blank
				        % line after name      

\address{\bf  \small PRESENT ADDRESS\\ 6 square Bolivar\\75019, Paris\\France\\+33 (0)6 16 87 56 12}
\address{\bf \small PERMANENT ADDRESS \\ Via Forlani 28a \\  31032, Casale sul Sile, TV \\ Italy}
                                  
\begin{resume}

\section{\small RESEARCH INTERESTS}  
    \vspace{5pt}	
    Primary: Labor Economics, Applied Micro-Econometrics. \\
    Others: Evaluation of Public Policies. 
    
\section{\small REFERENCES} \\  
\vspace{5pt}
 \begin{tabular}{@{}p{3.5in}p{3.5in}}
\href{https://sites.google.com/site/jmarcrobin/home}{Jean-Marc Robin}
& \href{https://sites.google.com/site/qaquocanhdo/}{Quoc-Anh Do} &   
Professor & Associate Professor \\
	Sciences Po  & Sciences Po
	  \\
	  Department of Economics & Department of Economics \\

 \phone\hspace{0.1cm}+33 (0)1 45 49 72 43 & \phone\hspace{0.1cm}+33 (0)1 45 49 83 58  \\
\Letter\hspace{0.1cm}\href{mailto:jeanmarc.robin@sciencespo.fr}{jeanmarc.robin@sciencespo.fr} &
\Letter\hspace{0.1cm}\href{mailto:quocanh.do@sciencespo.fr}{quocanh.do@sciencespo.fr}  
\end{tabular}

\vspace{0.25cm}

\begin{tabular}{@{}p{3.5in}p{3.5in}}
\href{https://floswald.github.io/}{Florian Oswald} \\
Assistant Professor \\
%Department of Economics & International Directorate\\
Sciences Po \\
Department of Economics \\

\phone\hspace{0.1cm}+33 (0)1 45 49 64 53 \\
\Letter\hspace{0.1cm}\href{mailto:florian.oswald@gmail.com}{florian.oswald@gmail.com} & \\
\end{tabular}
\section{\small EDUCATION} 
    \vspace{5pt}	
    \textit{Sciences Po, Paris, France} \\
    Ph.D. in Economics, September 2015 - 2019 (Expected). Supervisor: Prof. Jean-Marc Robin.\\
    M.Sc. in Economics, Sep 2013 - Jul 2015. Summa cum laude (top 2\%).\\
    \\[-2pt]        
    \textit{Universit\`{a} degli Studi di Padova, Italy}  \\       
    Bachelor in Economics and Management, Sep 2010 - Jul 2013. 110/110 with honours.\\
    Erasmus Exchange, \textit{Nova SBE, Lisbon}, Sep 2012 - Jan 2013.

\section{\small WORK IN PROGRESS}
    \vspace{5pt}	
    \textbf{Firm Dynamics and individual Earnings: Evidence from French Data}.
\section{\small WORKING PAPERS}
   \vspace{5pt}
\textbf{Payroll Taxation and the Cyclical Distribution of Income Shocks}, with Julien Pascal. \\
\\
 \textit{Abstract:} We develop a theoretical framework to evaluate the contribution of different payroll tax schedules to the cyclical behavior of the distribution of individual income shocks along the business cycle. We build a dynamic search-and-matching model of the labor market featuring heterogeneous workers, aggregate and idiosyncratic shocks and a non-linear payroll tax schedule. We solve the model using perturbation techniques developed in Reiter (2009). We estimate the model on Italian administrative data for the period 1980-2012 and use our estimated framework to quantitatively evaluate how different payroll tax schedules can alter the cyclicality of income risk for different types of workers. \\
\\
\textbf{Intra-Firm Hierarchies and Gender Gaps}, with Aseem Patel and Joanne Tan. \\
\\   
\textit{Abstract:} Within-firm factors are known to play a major role in wage inequality. This paper documents the role of hierarchies in gender wage and employment gaps using administrative French data. First, we document the extent and evolution of gender wage gaps across pre-defined layers of hierarchy within firms over time. We find that gender wage and representation gaps are larger in upper layers of hierarchy, even as they narrow over time. Second, we exploit a policy on corporate board quotas in France to study the impact of an increase in female leadership on gender wage and employment outcomes within firms. We find that a rise in the share of women on corporate boards narrows the gender wage gap only for top layers of hierarchy, but not for lower layers. We also observe that greater female corporate board membership increases the share of women working part-time in lower layers of hierarchy, at the expense of full-time work. The opposite effect holds for upper layers. The results suggest that, at least in the short term, female corporate board members improve the relative wages of women in close hierarchical proximity. They may also be able to affect change on flexi-work culture for the firm as a whole. However, due to the nonlinear returns to hours worked for upper layer professions, only lower layer women shift into part-time work. \\
\\
\textbf{How Social Networks Shape Our Beliefs: A Natural Experiment among Future French Politicians}, with Y. Algan, A. Le Chapelain, Q-A. Do and Y. Zenou. \\
\\
\textit{Abstract:} This paper shows how friendship shapes beliefs and political opinion. We make use of a unique natural experiment that randomly assigns new
freshmen into \integration groups" in a college that produces most of
France's top politicians. As pairs of students in the same integration
group are more likely to become friends. This same-group membership
thus serves as instrumental variable to estimate the effect of friendship
in dyadic regressions. We find strong, robust effects of friendship on
differences in beliefs after six months: becoming friends reduces the difference
by half a point on a ten-point scale of political opinions. It works
mostly by keeping friends from diverging, rather than pulling their opinions
closer. The effect becomes insignificant for second-degree friends.
The friendship effect completely dominates the effect of belonging to the
same tutorial group throughout the freshman year. The findings highlight
the importance of analyzing elicited friendship data instead of using
peer groups.
  
\section{\small TEACHING EXPERIENCE}
   \vspace{5pt}	
   \textit{Science Po, Paris} \\
   Graduate Econometrics 1: Statistics and Probability, Teaching Assistant, 2015; 2016; 2017; 2018. \\
   Introduction to Econometrics (Undergraduate), Lecturer, 2018. \\
   Econometrics: Evaluation of Public Policies (Graduate), Evaluator, Fall 2015. \\
   Macroeconomics of Development (Graduate), Evaluator, Spring 2015; 2016; 2017.

\section{\small RESEARCH EXPERIENCE}
   \vspace{5pt}
   Research Assistant, Sciences Po, June 2014 - December 2015.
    
\section{\small SCHOLARSHIPS AND AWARDS}
   \vspace{5pt}
Doctoral School Scholarship, Sciences Po, 2015 - 2018. \\
Summa cum Laude (top 2\%), Sciences Po, 2013 - 2015. \\
University Honours, Universit\`{a} degli Studi di Padova, 2010 - 2013.

\section{\small CONFERENCE AND SEMINARS PRESENTATIONS}
 \vspace{5pt}
 \begin{tabular}{@{}p{0.6in}p{5.7in}}
    2018: & Econometric Society Winter Meeting (scheduled); Annual Search and Matching Conference - Cambridge (Poster); Sciences Po Lunch Seminar. \\
	2017: & Sciences Po Lunch Seminar.
\end{tabular}
	 \vspace{5pt}

\section{\small COMPUTER SKILLS}
   \vspace{5pt}
	Julia, Python, R, Stata.

\section{\small LANGUAGES}
   \vspace{5pt}
    Italian (Native), English (Fluent), French (Fluent), Spanish (Beginner).
    
\end{resume}
\end{document}